%! Author Crfews
%! Date 7/30/2020

%Preamble
\documentclass[12pt]{article}
\setlength{\parindent}{0em}
\linespread{1.6}
\hbadness=99999
\hfuzz=9999pt
\title{Spring Quarantine Notes}
\author{Crfews}

%Packages
\usepackage{comment}
\usepackage{amsmath}
\usepackage{commath}
\usepackage{textcomp}
\usepackage{gensymb}
%Document
\begin{document}
    \maketitle
    \section{Limits}
    \subsection{Notation}
        $$\lim_{x\to\infty} f(x)$$
    \subsection{Formal Definition of a Limit}
        $$\lim_{x\to a} f(x) = L \iff \forall \ \varepsilon > 0, \exists \  \delta > 0  \ni\  if \  x \in N_{\delta}(a),\  then\  f(x)\in N_{\varepsilon}(L)$$
    \subsection{Examples}
        \(  \lim_{x\to 2} g(x) =
        \begin{cases}
                x^{2} &, x\neq 2\\
                1 &, x = 2\\
            \end{cases}
        \) $ = 4$
    \subsection{Determining Limits using the Squeeze Theorem}
        $f(x) \leq g(x) \leq h(x)$\newline
        $\lim_{x\to c} f(x) = L$ and $\lim_{x\to c} h(x) = L$ $\implies$ $\lim_{x\to c} g(x) = L$
    

\end{document}