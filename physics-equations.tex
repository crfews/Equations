%! Author = Crfews
%! Date = 7/29/2020

% Preamble
\documentclass[11pt]{article}
\setlength{\parindent}{0em}
\hbadness=99999
\hfuzz=9999pt

% Packages
\usepackage{amsmath}
\usepackage{commath}
\usepackage{textcomp}
\usepackage{gensymb}

% Document
\begin{document}
    \title{Physics 1 and 2 Reference Sheet}
    \author{Crfews}
    \maketitle

    \section{Mechanics}
    Final Velocity $v_{x} = v_{x0} + a_{x}t$\newline
    Position $x = x_{0} + v_{x0}t+\frac{1}{2}a_{x}t^{2}$\newline
    Final Velocity $v^{2}_{x} = v^{2}_{x0}t + \frac{1}{2}a_{x}t^{2}$\newline
    Acceleration $\vec{a} = \frac{\Sigma\vec{F}}{m} = \frac{\vec{F}_{net}}{m}$ \newline
    Force of Friction $\abs{\vec{F}_{f}} \leq \mu\abs{\vec{F}_{n}}$\newline
    Angular Acceleration $a_{c} = \frac{v^{2}}{r}$\newline
    Momentum $\vec{p} = m\vec{v}$\newline
    Impulse $\vec{p} = \vec{F}\Delta t$\newline
    Change in Energy $\Delta E = W = F\parallel d = Fd\cos{\theta}$\newline
    Power $P = \frac{\Delta E}{\Delta t}$\newline
    Angle $\theta = \theta_{0} + \omega_{0}t + \frac{1}{2}\alpha t^{2}$\newline
    Angular Speed $\omega = \omega_{0} + \alpha t$\newline
    Position $x = A\cos{\omega t} = A\cos{2\pi ft}$\newline
    Position $x_{cm} = \frac{\Sigma m_{i}x_{i}}{\Sigma m_{i}}$\newline
    Angular Acceleration $\alpha = \frac{\Sigma\vec{\tau}}{I} = \frac{\vec{\tau}}{I}$\newline
    Torque $\tau = r\perp F = rF\sin{\theta}$\newline
    Angular Momentum $L = I\omega$\newline
    Angular Momentum $L = \tau \Delta t$\newline

    \subsection{Energy and Simple Harmonic Motion}
    Kinetic Energy $K = \frac{1}{2}I\omega^{2}$\newline
    Force of Spring $\abs{\vec{F}_{s}} = k\abs{\vec{x}}$\newline
    Elastic Potential Energy $U_{s} = \frac{1}{2}kx^{2}$\newline
    Gravitational Potential Energy $\Delta U_{g} = mg\Delta y$\newline
    Period $T = \frac{2\pi}{\omega} = \frac{1}{2}$\newline
    Spring Period $T = 2\pi \sqrt{\frac{m}{k}}$\newline
    Pendulum Period $T_{p}=2\pi\sqrt{\frac{l}{g}}$\newline
    Force of Gravity $\abs {\vec{F}_{g}}= G \frac {m_{1}m_{2}}{r^{2}}$\newline
    Gravitational Acceleration $\vec{g} = \frac{\vec{F}_{g}}{m}$\newline
    Gravitational Potential Energy = $U_{G} = \frac{Gm_{1}m_{2}}{r}$\newline


    \section{Electricity and Magnetism}
    Coulomb's Law $\abs{\vec{F}_{E}} = \frac{1}{4\pi \epsilon_{0}}\abs{\frac{q_{1}q_{2}}{r^{2}}} = k\abs{\frac{q_{1}q_{2}}{r^{2}}}$\newline
    Electric Field by Point Charge $\vec{E} = \frac{\vec{F}_{E}}{q}$\newline
    Magnitude of Electric Field over Distance $\abs{\vec{E}} = \frac{1}{4\pi \epsilon_{0}}\frac{\abs{q}}{r^{2}}$\newline
    Electric Potential Energy $\Delta U_{E} = q\Delta{V}$\newline
    Voltage at Distance from Point Charge $V = \frac{1}{4\pi \epsilon_{0}}\frac{q}{r}$\newline
    Magnitude of Electric Field at Distance $\abs{\vec{E}} = \abs{\frac{\Delta V}{\Delta r}}$\newline
    Potential Difference of Capacitor $\Delta V = \frac{Q}{C}$\newline
    Capacitance $C = \kappa \epsilon_{0} \frac{A}{d}$\newline
    Electric Field of Capacitor $\frac{Q}{\epsilon_{0}A}$\newline
    Potential Energy of Capacitor $U_{C} = \frac{1}{2}Q\Delta V = \frac{1}{2}C(\Delta V)^{2}$\newline
    Current is rate of Charge over Time $I = \frac{\Delta Q}{\Delta t}$\newline
    Current $I = \frac{\Delta V}{R}$\newline
    Resistance $R = \frac{\rho l}{A}$\newline
    Current $I = \frac{\Delta V}{R}$\newline
    Power $P = I\Delta V$\newline
    Resistors in Series $R_{s} = \sum{i}{}R_{i}$\newline
    Resistors in Parallel $\frac{1}{R_{p}} = \sum_{i}^{}\frac{1}{R_{i}}$\newline
    Capacitors in Parallel $C_{p} = \sum_{i}^{}C_{i}$\newline
    Capacitors in Series $\frac{1}{C_{s}} = \sum{i}^{}\frac{1}{C_{i}}$\newline
    Magnetic Field $B = \frac{\mu_0}{2\pi}\frac{I}{r}$\newline
    Magnetic Force by Field on a Point Charge $\vec{F}_{M} = q\vec{v} \times \vec{B}$\newline
    Magnetic Force by Field at Angle on a Point Charge $\abs{\vec{F}_{M}} = \abs{q\vec{v}}\abs{\sin{\theta}}\abs{\vec{B}}$\newline
    Magnetic Force by Wire with Known Length and/or Field $\vec{F}_{M} = I\vec{l} \times \vec{B}$\newline
    Magnetic Force by Wire at Angle $\abs{\vec{F}_{M}} = \abs{I\vec{l}}\abs{\sin{\theta}}\abs{\vec{B}}$\newline
    Magnetic Flux on Wire Loop $\Phi_{B} = \vec{B}\bullet\vec{A}$\newline
    Magnetic Flux on Wire Loop at Angle to Magnetic Field $\abs{\Phi_{B}} = \abs{\vec{B}}\cos{\theta}\abs{\vec{A}}$\newline
    Electromotive Force (emf) is Magnetic Flux over Time $\varepsilon = -\frac{\Delta\Phi_{B}}{\Delta t}$\newline
    emf on Length of Wire Moving through a Magnetic Field $\varepsilon = Blv$


    \section{Fluid Mechanics and Thermodynamics}
    Density $\rho = \frac{m}{V}$\newline
    Pressure is Force over Area $\frac{F}{A}$\newline
    Total Pressure $P = P_{0} + \rho gh$\newline
    Archimedes' Principle $F_{b} = \rho Vg$\newline
    Flow Rate is Constant $A_{1}v{1} = A_{2}v_{2}$\newline
    Bernoulli's Equation $P_{1} + \rho gy_{1} + \frac{1}{2}\rho v_{1}^{2} = P_{2} + \rho gy_{2} + \frac{1}{2}\rho v_{2}^{2}$\newline
    Thermal Conductivity $\frac{Q}{\Delta t} = \frac{kA\Delta T}{L}$\newline
    Ideal Gas Law $PV=nRT=Nk_{B}T$\newline
    Kinetic Energy of an Ideal Gas $K = \frac{3}{2}k_{B}T$\newline
    Work Done On Gas By System $W = -P\Delta V$\newline
    Internal Energy of a Gas $\Delta U = Q + W$\newline \newline
    \begin{tabular}{|c|c|}
        \hline
        \multicolumn{2}{|c|}{Formulas}\\
        \hline
        Density & $\rho = \frac{m}{V}$\\
        \hline
        Definition of Pressure & $\frac{F}{A}$\\
        \hline
        Total Pressure & $P = P_{0} + \rho gh$\\
        \hline
        Archimedes' Principle & $F_{b} = \rho Vg$\\
        \hline
        Flow Rate & $A_{1}v_{1}=A_{2}v_{2}$\\
        \hline
        Bernoulli's Equation & $P_{1} + \rho gy_{1} + \frac{1}{2}\rho v_{1}^{2} = P_{2} + \rho gy_{1} + \frac{1}{2}\rho v_{2}^{2}$\\
        \hline
        Thermal Conductivity & $\frac{Q}{\Delta t} = \frac{kA\Delta T}{L}$\\
        \hline
        Ideal Gas Law & $PV=nRT=Nk_{B}T$\\
        \hline
        Kinetic Energy of an Ideal Gas & $K = \frac{3}{2}k_{B}T$\\
        \hline
        Work Done on a Gas by the System & $W = -P\Delta V$\\
        \hline
        Internal Energy of a Gas & $\Delta U = Q + W$\\
        \hline
    \end{tabular}
    

    \section{Modern Physics}
    Energy of a Photon $E = hf$\newline
    Photoelectric Effect $K_{max} = hf - \varphi$\newline
    de Broglie Wavelength $\lambda = \frac{h}{p}$\newline
    Mass-energy Equivalence $E = mc^2$


    \section{Waves and Optics}
    Wavelength $\lambda = \frac{v}{f}$\newline
    Index of Refraction $n = \frac{c}{v}$\newline
    Snell's Law $n_{1}\sin{\theta_{1}} = n_{2}\sin{\theta_{2}}$\newline
    Distance from Image and/or Object to Focus $\frac{1}{s_{i}} + \frac{1}{s_{0}} = \frac{1}{f}$\newline
    Magnification of an Image $\abs{M} = \abs{\frac{h_{i}}{h_0}} = \abs{\frac{s_{i}}{s_{0}}}$
    \subsection{Young's Double Slit Experiment}
        Distance between Nodes/Antinodes $\Delta L = m\lambda$\newline
        Angle between Nodes/Antinodes $d\sin{\theta} = m\lambda$
        


    \section{Geometry and Trigonometry}
    \subsection{Two-Dimensional Figures}
    \subsubsection{Rectangle}
    Area $A = bh$
    \subsubsection{Triangle}
    Area $A = \frac{1}{2}bh$
    \subsubsection{Right triangle}
    Pythagorean Theorem $c^{2} = a^{2} + b^{2}$\newline
    Sine $\sin{\theta} = \frac {a}{c}$\newline
    Cosine $\cos{\theta} = \frac{b}{c}$\newline
    Tangent $\tan{\theta} = \frac{a}{b}$
    \subsubsection{Circle}
    Area $A = \pi r^{2}$\newline
    Circumference $C = 2\pi r$
    \subsection{Three-Dimensional Figures}
    \subsubsection{Rectangular solid}
    Volume $V = lwh$
    \subsubsection{Cylinder}
    Volume $V = \pi r^{2}l$\newline
    Surface Area $S = 2\pi rl + 2\pi r^{2}$
    \subsubsection{Sphere}
    Volume $V = \frac{4}{3}\pi r^{3}$\newline
    Surface Area $S = 4 \pi r^{2}$


    \section{Constants and Conversion Factors}
    Proton mass, $m_{p} = 1.67 \times 10^{-27} kg$\newline
    Neutron mass, $m_{n} = 1.67 \times 10^{-27} kg$\newline
    Electron mass, $m_e = 9.11 \times 10^{-31} kg$\newline
    Avogadro's number, $N_{0} = 6.02 \times 10^{23} mol^{-1}$\newline
    Universal gas constant, $R = 8.31 J/(mol\bullet K)$\newline
    Boltzmann's constant, $k_{B} = 1.38 \times 10^{-23} J/K$\newline
    1 electron volt, $1 eV = 1.60 \times 10^{-19} J$\newline
    Speed of light, $c = 3.00 \times 10^{8} m/s$\newline
    Electron charge magnitude, $e = 1.60 \times 10^{-19} C$\newline
    Coulomb's law constant, $k = \frac{1}{4\pi \epsilon_{0}} = 9.0 \times 10^{9}$ $N\bullet m^{2} / C^{2}$\newline
    Universal gravitational constant, $G = 6.67 \times 10^{-11}$ $m^{3}/kh\bullet s^{2}$\newline
    Acceleration due to gravity at Earth's surface, $g = 9.8 m/s^{2}$\newline
    1 unified atomic mass unit $1 u = 1.66 \times 10^{-27} kg = 931 MeV/c^{2}$\newline
    Planck's constant, $h = 6.63 \times 10^{-34} J\bullet s = 4.14 \times 10^{-15} eV\bullet s$\newline
    $hc = 1.99 \times 10^{-25} J/m = 1.24 \times 10^{3} eV\bullet nm$\newline
    Vacuum permittivity, $\varepsilon_{0} = 8.85 \times 10^{-12} C^{2}/N\bullet m^{2}$\newline
    Coulomb's law constant, $k = \frac{1}{4\pi \epsilon_{0}} = 9.0 \times 10^{9}$\newline
    Vacuum permeability, $\mu_{0} = 4\pi \times 10^{-7} (T\bullet m)/A$\newline
    Magnetic constant, $k' = \frac{\mu_{0}}{4\pi} = 1 \times 10^{-7} (T\bullet m)/A$\newline
    1 atmosphere pressure, $1 atm = 1.0 \times 10^{5} N/m^{2} = 1.0 \times 10^{5} Pa$


    \section{Unit Symbols}
    \begin{flushright}
        \begin{tabular}{|c|c|c|}
            \hline
            Quantity & Unit & Abbreviation\\
            \hline
            Length & meter & m\\
            \hline
            Mass & kilogram & kg\\
            \hline
            Time & second & s\\
            \hline
            Amperage & ampere & A\\
            \hline
            Temperature & kelvin & K\\
            \hline
            Count & mole & mol\\
            \hline
            Cycle & hertz & Hz\\
            \hline
            Force & newton & N\\
            \hline
            Pressure & pascal & Pa\\
            \hline
            Energy & joule & J\\
            \hline
            Power & watt & W\\
            \hline
            Charge & coulomb & C\\
            \hline
            Potential Difference & volt & V\\
            \hline
            Resistance & ohm & $\Omega$\\
            \hline
            Inductance & henry & H\\
            \hline
            Capacitance & farad & F\\
            \hline
            Magnetic Flux/Induction & tesla & T\\
            \hline
            Temperature & degree Celsius & \degree C\\
            \hline
            Energy & electron volt & eV\\
            \hline
        \end{tabular}
    \end{flushright}


    \section{Prefixes}
    \begin{flushright}
        \begin{tabular}{|c|c|c|}
            \hline
            Factor & Prefix & Symbol\\
            \hline
            $10^{12}$ & tera & T\\
            \hline
            $10^{9}$ & giga & G\\
            \hline
            $10^{6}$ & mega & M\\
            \hline
            $10^{3}$ & kilo & k\\
            \hline
            $10^{-2}$ & centi & c\\
            \hline
            $10^{-3}$ & milli & m\\
            \hline
            $10^{-6}$ & micro & $\mu$\\
            \hline
            $10^{-9}$ & nano & n\\
            \hline
            $10^{-12}$ & pico & p\\
            \hline
        \end{tabular}
    \end{flushright}

    \section{Angles}
    \begin{flushright}
        \begin{tabular}{|c|c|c|c|c|c|c|c|}
            \hline
            \multicolumn{8}{|c|}{Values of Trigonometric Functions For Common Angles}\\
            \hline
            $\theta$ & 0\degree & 30\degree & 37\degree & 45\degree & 53\degree & 60\degree & 90\degree\\
            \hline
            $\sin{\theta}$ & $0$ & $\frac{1}{2}$ & $\frac{3}{5}$ & $\frac{\sqrt{2}}{2}$ & $\frac{4}{5}$ & $\frac{\sqrt{3}}{2}$ & $1$\\
            \hline
            $\cos{\theta}$ & $1$ & $\frac{\sqrt{3}}{2}$ & $\frac{4}{5}$ & $\frac{\sqrt{2}}{2}$ & $\frac{3}{5}$ & $\frac{1}{2}$ & $0$\\
            \hline
            $\tan{\theta}$ & $0$ & $\frac{\sqrt{3}}{3}$ & $\frac{3}{4}$ & $1$ & $\frac{4}{3}$ & $\sqrt{3}$ & $\infty$\\
            \hline
        \end{tabular}
    \end{flushright}

\end{document}
